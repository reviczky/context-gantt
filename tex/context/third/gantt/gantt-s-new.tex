%D \module
%D   [     file=t-gantt,
%D      version=2011.09.24,
%D        title=\CONTEXT\ User Module,
%D     subtitle=Drawing Gantt charts,
%D       author={Adam Reviczky and Aditya Mahajan},
%D         date=\currentdate,
%D    copyright={Adam Reviczky and Aditya Mahajan},
%D      license=Public Domain]

\writestatus{loading}{Gantt charts / MetaPost backend (ver: 2011.09.24)}

\startmodule[gantt-s-mp]
\unprotect

\enabletrackers[interfaces.namespaces,context.flush]

\definenamespace
  [ganttchart]
  [
    \c!type=module,
    \c!name=ganttchart,
    setup=\v!yes,
  ]

\definenamespace
  [gantttask]
  [
    \c!type=module,
    \c!name=gantttask,
    \c!style=\v!yes,
    \c!command=\v!yes,
    setup=\v!list,
  ]

\appendtoks
  \ifinsideganttchart
      \appendtocommalist \currentgantttask \current_gantt_tasks
  \else
      % Ideally, we don't need this, but something weird is happening
      \setupgantttask[\currentgantttask][parent=\????gantttask]
  \fi
\to \everydefinegantttask

\starttexdefinition install_gantt_task_handler #1#2
    %% #1 =  \definegantttask[##1[##2][alternative=#2, ##3]
    \def#1{\doquadrupleargument\redodefinegantttask[#2]}
\stoptexdefinition

\starttexdefinition redodefinegantttask [#1][#2][#3][#4]
  \iffourthargument
    \definegantttask[#2][\s!parent=\????gantttask#3,\c!alternative=#1, #4]
  \else
    \definegantttask[#2][\s!parent=\????gantttask,\c!alternative=#1, #3]
  \fi
\stoptexdefinition

\install_gantt_task_handler \define_gantt_task      {task}
\install_gantt_task_handler \define_gantt_milestone {milestone}

\def\current_gantt_tasks{}

\startsetups ganttchart::setups
  \def\current_gantt_tasks{}

  \let\task         \define_gantt_task
  \let\milestone    \define_gantt_milestone
\stopsetups

\newif\ifinsideganttchart

\def\startganttchart
    {\dosingleargument\start_gantt_chart_indeed}

\starttexdefinition start_gantt_chart_indeed [#1]
    \begingroup
    \insideganttcharttrue
    \setupganttchart[#1]
    \processcommacommand[ganttchart::setups,\ganttchartparameter\c!setups]\directsetup
\stoptexdefinition

\starttexdefinition stopganttchart
    \writestatus{DEBUG}{\current_gantt_tasks}
    \initialize_gantt_chart_display
    \processcommacommand[\current_gantt_tasks]\add_gantt_chart_task
    \show_gantt_chart_display
    \endgroup
    \insideganttchartfalse
\stoptexdefinition

\newcount\gantt_task_count

\starttexdefinition initialize_gantt_chart_display
   \gantt_task_count=0
   \resetMPdrawing
   \startMPdrawing
     save gantt_bar_size, gantt_bar_shift ;
     pair gantt_bar_size, gantt_bar_shift ;
      
     save gantt_square ; path gantt_square ;
     gantt_square := (0,-0.5) -- (1,-0.5) -- (1,0.5) -- (0,0.5) -- cycle ;
   \stopMPdrawing
\stoptexdefinition

\starttexdefinition add_gantt_chart_task #1
  \increment\gantt_task_count 
  \edef\currentgantttask{#1}
  \startMPdrawing
     gantt_bar_shift :=
        ( \gantttaskparameter\c!start * \gantttaskparameter\c!x, 
         -\gantt_task_count * \gantttaskparameter\c!y ) ;

     gantt_bar_size := 
        ( (\gantttaskparameter\c!stop - \gantttaskparameter\c!start)*\gantttaskparameter\c!x,
          0.5*\gantttaskparameter\c!y ) ;

  \stopMPdrawing
  \getvalue{gantt_draw_\gantttaskparameter\c!alternative}{#1}
  \startMPdrawing
     %% Draw the label
     %% \doset...attributes is defined as an unexpandable macro. Therefore, it
     %$ uses the value of \type{\currenttask} when it is called. To avoid that
     %% we also set the value of \type{\currenttask}. However, the contents of
     %% textext is expanded twice, so we need to ensure that the write string is 
     %% created after the second expansion
     label.lft (textext("\def\noexpand\letterbackslash currentgantttask{\currentgantttask}
        \dosetgantttaskattributes{\c!style}{\c!color} \gantttaskparameter\c!text"),
       (-\gantttaskparameter\c!distance, ypart gantt_bar_shift)) ;
  \stopMPdrawing
\stoptexdefinition

\starttexdefinition gantt_draw_task #1
  \startMPdrawing
     %% Draw the completed task
     fill gantt_square %
        xyscaled (xpart gantt_bar_size * \gantttaskparameter{completed},
                  ypart gantt_bar_size) %
        shifted  gantt_bar_shift %
        withcolor \MPcolor{\gantttaskparameter{fillcolor}} ;

     %% Draw float
     if \gantttaskparameter{float} <> 0 :
           draw gantt_square %
              xyscaled (\gantttaskparameter{float}*\gantttaskparameter\c!x,
                        ypart gantt_bar_size) %
              shifted (gantt_bar_shift + (xpart gantt_bar_size, 0)) %
              withpen pencircle scaled \gantttaskparameter{floatthickness} %
              \gantttaskparameter{floatstyle} %
              withcolor \MPcolor{\gantttaskparameter{floatcolor}} ;
     fi ;
     %% Draw the whole task
     stripe_angle := 45 ;
     stripe_gap   := \gantttaskparameter\c!y / 5 ;
     stripe_path_a (withpen pencircle scaled \gantttaskparameter{stripethickness} withcolor \MPcolor{\gantttaskparameter{stripecolor}})
                   (draw)
      gantt_square % 
        xyscaled gantt_bar_size  %
        shifted  gantt_bar_shift %
        withpen pencircle scaled \gantttaskparameter{barthickness} %
        withcolor \MPcolor{\gantttaskparameter{barcolor}}
        ;
  \stopMPdrawing
\stoptexdefinition

\starttexdefinition gantt_draw_milestone #1
   \startMPdrawing
      fill \gantttaskparameter{milestoneshape} %
           xyscaled (ypart gantt_bar_size, ypart gantt_bar_size) %
           shifted  gantt_bar_shift %
           withcolor \MPcolor{\gantttaskparameter{milestonecolor}} ;
   \stopMPdrawing
\stoptexdefinition

\starttexdefinition show_gantt_chart_display
   \startMPdrawing
      %% Placeholder
   \stopMPdrawing
   \MPdrawingdonetrue\getMPdrawing
   \resetMPdrawing
\stoptexdefinition

\startmode[gantt-table] % Useful for debugging
%% Temporary method to show a gantt chart as a table
%% Will work on display as a metapost graphic

\newtoks \ganttchartdisplay

\starttexdefinition initialize_gantt_chart_display
  \ganttchartdisplay {}
  \appendtoks
    \NC Name \NC Type \NC Start \NC Stop \NC Float \NC Shape \NC \NR
  \to \ganttchartdisplay
\stoptexdefinition

\starttexdefinition add_gantt_chart_task #1
  \appendtoks
      \NC \namedgantttaskparameter{#1}\c!text
      \NC \namedgantttaskparameter{#1}\c!alternative
      \NC \namedgantttaskparameter{#1}\c!start
      \NC \namedgantttaskparameter{#1}\c!stop
      \NC \namedgantttaskparameter{#1}{float}
      \NC \namedgantttaskparameter{#1}{shape}
      \NC \NR
  \to \ganttchartdisplay
\stoptexdefinition

\starttexdefinition show_gantt_chart_display
    \startTABLE
      \the\ganttchartdisplay
    \stopTABLE
\stoptexdefinition
\stopmode

\setupganttchart
  [
    % \c!factor=1000, %Not sure why this is needed
    \c!style=\v!normal,
    \c!color=,
    \c!titlestyle=\v!small,
    \c!titlecolor=,
    \c!grid=\v!off, % \v!on, \v!horizontal, \v!vertical, \v!both
    \c!setups=,
  ]

\setupgantttask
  [
    \c!color=,
    \c!style=\bold,
    linewidth=1pt,
    % Don't change these
    completed=0, % From 0 to 1
    float=0,
    % Stripe
    stripethickness=0.5pt,
    stripecolor=lightblue,
    % Bar
    barthickness=1pt,
    barcolor=blue,
    % Fill
    fillcolor=blue,
    % Float
    floatcolor=darkgray,
    floatthickness=1pt,
    floatstyle={dashed evenly},
    % Mileston
    milestonecolor=green,
    milestoneshape=fulldiamond,
    % Dimensions
    \c!x=1cm,       
    \c!y=\the\dimexpr1.5\lineheight\relax,
    \c!distance=0.5cm,
    \c!stop=\gantttaskparameter\c!start,
  ]


\protect
\stopmodule

\starttext

\definegantttask
    [important]
    [
      color=red,
      style=\italic,
      barcolor=red,
      fillcolor=green,
    ]

\startganttchart
  \task      [one]                 [text={Task one}, start=0, stop=2, float=1, completed=0.5]
  \task      [two]   [important]   [text={Second},   start=2, stop=8, completed=0.25]
  \milestone [finish]              [shape=circle,    start=3]
\stopganttchart

\stoptext
